\part{Contexte}
Les stations de bases consomment 1\% de l’énergie totale produite.
Ce simple constat montre qu’une quantité très importante d’énergie est utilisée pour l’envoi d’ondes électromagnétiques.
Une grande partie de cette énergie est perdue car les stations de base fonctionnent en continu.
Ces stations émettent sur différentes fréquences car les applications sont multiples :
    téléphonie (2G, 3G, 4G),
    télévision (TNT),
    radio.
Certaines stations ont également des objectifs moins connus du grand public comme le guidage des avions pour l’atterrissage,
    le décollage et la communication sol-air.
Ainsi, de par ce projet, nous avons souhaité dans un premier temps observer le spectre électromagnétique.
Cette observation a plusieurs objectifs : visualiser les niveaux de puissance
    sur des fréquences allant de 100 MHz à 3GHz
    (couvrant ainsi toutes les applications vues précédemment)
    pour quantifier les niveaux maximums d’émissions mais également l’évolution au cours du temps.
Les mesures sont réalisées à ESIEE Paris et sur le toit de la société Ommic. Cette analyse a une visée énergétique.
En effet l’objectif final de celle-ci est de repérer les bandes d’intérêt ayant le plus d’énergie au cours du temps.
Notre travail est en lien avec celui réalisé par Vaclav Valenta en 2010.




Le second objectif de ce projet est de réaliser un système permettant
    la récupération de cette énergie « perdue » pour s’en resservir à des fins d’alimentation de systèmes électroniques.
    Afin de réaliser cela nous utiliserons les travaux de l’étude précédente pour dimensionner
    le module de conversion d’énergie sur les fréquences optimales déterminées.

\todo{A DVP Bryan/Maxime}

On voit donc la visée de ce projet permettant de faire fonctionner
    sans apport d’énergie de type batterie ou pile des
    systèmes électroniques dont les applications peuvent êtres très variées.


\newpage